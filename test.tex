\documentclass{article}
\usepackage{ctex}
\usepackage{tikz}
\usepackage{amsthm}
\usepackage{amsmath}
\usepackage{amssymb}
\usetikzlibrary{cd}
\usepackage{indentfirst}


%theorem,corollary,lemma
\newtheorem{theorem}{Theorem}[section]
\newtheorem{corollary}{Corollary}[theorem]
\newtheorem{lemma}[theorem]{Lemma}
\newtheorem{definition}[theorem]{Definition}
\newtheorem{example}[theorem]{Example}


\title{Category Theory}
\author{maple,菜鸡}
\date{\today}

\begin{document}
\maketitle

\section{Functoriality}

\begin{flushleft}
\textbf{函子}(functor)\\
\begin{definition}
一个函子$F \colon \mathcal{C} \rightarrow \mathcal{D}$由两个映射组成
\begin{itemize}
 \item 每一个对象$c \in \mathcal{C}$,对应一个对象$F_c \in \mathcal{D}$。
 \item 每一个态射$f \colon c \rightarrow	c' \in \mathcal{C}$,对应一个态射$Ff \colon Fc \rightarrow Fc' \in \mathcal{D}$,所有$Ff$的domain和codomian分别对应$f$的domain和codomain。 
\end{itemize}
\end{definition}
同时还满足两个公理(functoriality aximos)
\begin{itemize}
 	\item 对任意的可复合态射$f$,$g \in \mathcal{C}$,$Fg \cdot Ff = F(g \cdot f)$
 	\item 对任意的对象$c \in \mathcal{C}$,$F(1_c)=1_{F_c}$
\end{itemize}
通俗一点来说,函子由两个范畴之间对象的映射和态射的映射组成,同时保留了范畴所有的结构,包括domains,codomain,复合律,identity。
\end{flushleft}

\begin{lemma}
函子保留同构
\end{lemma}

\begin{flushleft}
\textbf{同变函子和逆变函子}(covariant and contravariant)\\
\begin{definition}
前面定义的函子也叫同变函子,那么一个逆变函子从$C$到$D$为$F: \mathcal{C}^{op} \rightarrow \mathcal{D} $,也是由两个映射组成
\begin{itemize}
	\item 每一个对象$c \in \mathcal{C}$,对应一个对象$F_c \in \mathcal{D}$。
	\item 每一个态射$f \colon c \rightarrow	c' \in \mathcal{C}$,对应一个态射$Ff \colon Fc' \rightarrow Fc \in \mathcal{D}$,所有$Ff$的domain和codomian分别对应$f$的codomain和domain
\end{itemize}
\end{definition}
同时也满足两个公理(functoriality aximos)\\
\begin{itemize}
	\item 对任意的可复合的$f$,$g \in \mathcal{C}$,$Ff \cdot Fg = F(g \cdot f)$
	\item 对任意的对象$c \in \mathcal{C}$,$F(1_c)=1_{F_c}$
\end{itemize}
相对于同变函子来说,$f$对应的像$Ff$的箭头反过来了,下面这个图可以很形象的描述	
\end{flushleft}

\begin{center}
$C^{op} \xrightarrow{F} \mathcal{D}$ \\
% https://tikzcd.yichuanshen.de/#N4Igdg9gJgpgziAXAbVABwnAlgFyxMJZABgBpiBdUkANwEMAbAVxiRAGMQBfU9TXfIRRkAjFVqMWbdgHJuvEBmx4CREeXH1mrRCABiAfU48+ywWtJjqWqbsPBZXbuJhQA5vCKgAZgCcIALZIZCA4EEgATNQ4dFgMbAF0aHBh8j7+QYjqoeGIAMzRsfG6icmpJiB+gcHRudk2OpVplRlIBTmR1pKNet7OXEA
\begin{tikzcd}
c \arrow[r, maps to] \arrow[d, "f"] & F_c                    \\
c' \arrow[r, maps to]               & F_{c'} \arrow[u, "Ff"]
\end{tikzcd}
\end{center}

\begin{flushleft}
\textbf{函子的复合}
\end{flushleft}
设$F \colon \mathcal{A} \rightarrow \mathcal{B},G \colon \mathcal{B} \rightarrow \mathcal{C}$是函子,定义$GF \colon \mathcal{A} \rightarrow \mathcal{C}$使得对$\mathcal{A}$中的任意一个对象$A$,\[A \mapsto G(F(A))\]
对$\mathcal{A}$中的任意一个态射$f \colon A \rightarrow B$, \[(f \colon A \rightarrow B) \mapsto (G(F(f)) \colon G(F(A)) \rightarrow G(F(b))) \]则GF是一个函子。

很自然地,因为函子可以进行复合运算,那么是否存在一个以所有范畴为对象,函子为态射的范畴? 但是遗憾的是两个范畴之间的函子的全体未必是一个集合。但是我们把目光限制在所有小范畴上时,我们的确可以得到一个以所有小范畴为对象,以小范畴之间函子为态射的范畴$\bf(Cat)$

\begin{definition}
设$F \colon \mathcal{C} \rightarrow \mathcal{D}$是一个函子。如果存在函子$G \colon \mathcal{D} \rightarrow \mathcal{C}$使得$GF=1_{\mathcal{C}},FG = 1_{\mathcal{D}}$,则称$F$是范畴$\mathcal{C}$到范畴$\mathcal{D}$的一个$\textbf{同构}$(isomorphism)。换一句话来说如果存在范畴$\mathcal{C}$到$\mathcal{D}$的同构$F \colon \mathcal{C} \rightarrow \mathcal{D}$,则称范畴$\mathcal{C}$与$\mathcal{D}$是同构的。
\end{definition}



\begin{flushleft}
\textbf{c-Functor} \\
%待补充
\begin{center}
$\mathcal{C} \xrightarrow{C(c,-)} \mathcal{S}et$ \\
$\mathcal{C}^{op} \xrightarrow{C(-,c)} \mathcal{S}et$ \\
\end{center}
\end{flushleft}


\begin{flushleft}
$\textbf{Product}$ \\
\begin{definition}
对任意的两个范畴 $\mathcal{C}$ 和 $\mathcal{D}$,他们的积是一个新范畴 $\mathcal{C} \times \mathcal{D}$\\
\begin{itemize}
	\item 对象是有序对 $\left( c,d \right)$,其中$c$是$\mathcal{C}$中的一个对象,$d$是是$\mathcal{D}$的一个对象 
	\item 态射也是有序对 $\left(f,g \right) \colon \left(c,d\right) \rightarrow \left(c',d'\right)$,$f\colon c \rightarrow c' \in \mathcal{C}$,$g \colon d \rightarrow d' \in \mathcal{D}$
\end{itemize}
\end{definition}
\end{flushleft}


\begin{flushleft}
\textbf{双函子的形式化定义}(bifunctor) \\ 
\begin{definition}
%https://encyclopediaofmath.org/wiki/Bifunctor
它叫双函子,不如叫二元函子,因为bifunctor是binary functor的缩写,对范畴$\mathcal{C}_1,\mathcal{C}_2,\mathcal{D}$,二元函子表示为:\\

\begin{center}
$F \colon \mathcal{C}_1 \times \mathcal{C}_2 \rightarrow \mathcal{D}$
\end{center}
\end{definition}

这个函子的domain是两个范畴的积,例如Abel群看做一个范畴$\mathbf{G}$,其中加法运算 $+ \colon \mathbf{G} \times \mathbf{G} \rightarrow \mathbf{G}$是一个双函子(证明过程证明functor满足的两个公理即可,保持identity和复合律)。 

\end{flushleft}

\pagebreak
\begin{flushleft}
$\textbf{自然双函子}$,好吧它的真正名字叫$\textbf{射影函子}$ \\
\begin{center}
$P_{\mathcal{C}} \colon \mathcal{C} \times \mathcal{D} \rightarrow \mathcal{C},\left( A,B \right)\mapsto A,\left( f,g \right)\mapsto f$ \\ \vbox{}
$P_{\mathcal{D}} \colon \mathcal{C} \times \mathcal{D} \rightarrow \mathcal{D},\left( A,B \right)\mapsto B,\left( f,g \right)\mapsto g$ \\
\end{center} 
\end{flushleft}




\begin{flushleft}
\textbf{射影函子的万有性质} \\
对任意的范畴 $\varepsilon$ 及函子 $R\colon \varepsilon \rightarrow \mathcal{C}$    和 $T\colon \varepsilon \rightarrow \mathcal{D}$ ,存在唯一的函子 $F\colon \varepsilon \rightarrow \mathcal{C}\times\mathcal{D}$ 使得 $P_{\mathcal{C}}F=R,P_{\mathcal{D}}F=T$, 即下面交换图表示
\end{flushleft}

% https://tikzcd.yichuanshen.de/#N4Igdg9gJgpgziAXAbVABwnAlgFyxMJZABgBpiBdUkANwEMAbAVxiRAB12BbOnACwDGjYAGEAvgAJOeLvCndeg4QBExIMaXSZc+QigBM5KrUYs2nHvyENRajVux4CRQ-uP1mrRB3b0ATjBo2AzO9iAYjrpEZG7UHmbeForWwKrqxjBQAObwRKAAZn4QXEhkIDgQSACM1Ax0AEYwDAAK2k56IAww+TggcaZeIM0A+sBJVsLidpoghcWl1BVIAMz9nmwjYwoTNqrTBUUliIbllYirJuveACrqM3NHJ0uINZcJIABKdwfzx4tnZQYWDAgygdDgfEyfTegwAYukxEA
\begin{center}
\begin{tikzcd}
\mathcal{C} \times \mathcal{D} \arrow[rr, "P_{\mathcal{C}}"] \arrow[dd, "P_{\mathcal{D}}"] &  & \mathcal{C}                                                           \\
                                                                                           &  &                                                                       \\
\mathcal{D}                                                                                &  & \varepsilon \arrow[ll, "T"] \arrow[uu, "R"] \arrow[lluu, "F", dashed]
\end{tikzcd}
\end{center}

\section{Naturality}
\begin{definition}

\textbf{自然变换}(natural transformation)
设$\mathcal{C}$与$\mathcal{D}$是两个范畴,$F \colon \mathcal{C}\rightarrow \mathcal{D}$与$G \colon \mathcal{C} \rightarrow \mathcal{D}$,一个自然变换下面箭头组成
\begin{itemize}
	\item 一个箭头 $\alpha_c \colon F_c \rightarrow G_c$表示对于每一个对象$c \in \mathcal{C}$,这些箭头的“collection”(注意没有用集合的概念)定义了自然变换的组成部分。
\end{itemize}
所有任意的态射$f \colon c \rightarrow c' \in \mathcal{C}$,下面图交换($G(f)\alpha_{c}=\alpha_{c'}F(f)$):
\begin{center}
% https://tikzcd.yichuanshen.de/#N4Igdg9gJgpgziAXAbVABwnAlgFyxMJZABgBpiBdUkANwEMAbAVxiRADEB9AYxAF9S6TLnyEUARnJVajFmwDiPfoJAZseAkTLjp9Zq0QdOwbgHI+yoetFFJO6nrmHFJ8-2kwoAc3hFQAMwAnCABbJDIQHAgkSRl9NgAdBMY0AAs6JQEA4LDECKikACZqBjoAIxgGAAVhDTEQQKwvVJwQB1kDDn9LECDQmOoCxABmdvjnbqzenKLB6JGS8sqa601DRubWsacQDNcLPgo+IA
\begin{tikzcd}
F_c \arrow[r, "\alpha_c"] \arrow[d, "Ff"'] & G_c \arrow[d, "Gf"] \\
F_{c'} \arrow[r, "a_{c'}"']                & G_{c'}             
\end{tikzcd}
\end{center}

\end{definition}
注意交换图里面的所有态射都属于$\mathcal{D}$,如果自然变换$\alpha \colon F \rightarrow G$满足对任意的$c \in ob\mathcal{C} , \alpha_{c}:F(A) \rightarrow G(A)$是一个同构,则称$\alpha$是$\textbf{自然同构}$(natural isomorphism) 

这个自然变换的定义看起来还是有一些抽象,如果在一个自然变换中,把对象$c$看成一个变量,$\alpha$的domain和codomain都可以用$c$来表示,而这些箭头是"collection"都是目标范畴的态射,所以定义可以形象的表示为
\begin{center}
$\alpha \colon ob\mathcal{C} \rightarrow Mor\mathcal{D}$
\end{center}

\pagebreak
\begin{lemma}
设$F,G,H \colon \mathcal{C} \rightarrow \mathcal{D}$和$T \colon \mathcal{D} \rightarrow \mathcal{C}$是函子,$\alpha \colon F \rightarrow G, \beta \colon G \rightarrow H$是自然变换,则
\begin{itemize}
 \item $\beta\alpha \colon F \rightarrow H \colon c \mapsto(F(c) \xrightarrow{\beta_c\alpha_c} H(c))$是一个自然变换
 \begin{center}
% https://tikzcd.yichuanshen.de/#N4Igdg9gJgpgziAXAbVABwnAlgFyxMJZABgBpiBdUkANwEMAbAVxiRADEB9AYxAF9S6TLnyEUARnJVajFmwDiPfoJAZseAkQBMU6vWatEIABJKBQ9aKJlx0-XKNdg3AOR9lFkZomlbe2YYgis5uHqrCGmLIOn4yBmymIe580jBQAObwRKAAZgBOEAC2SGQgOBBIknEOIAA6tYxoABZ0Zir5RZXU5Ug6IAx0AEYwDAAKEVZGDDA5OCD+8Ub1wzitvOYgHcWIfT2IAKwLNcY5YVtIACzdFQdHgcswq5xJZwXbVXtX1YHypxvniAAzNdLnc2PVGi1nq5ku03iUQUCwY4-hQ+EA
	\begin{tikzcd}
F_c \arrow[r, "\alpha_c"] \arrow[d, "Ff"] & G_c \arrow[r, "\beta_c"] \arrow[d, "Gf"] & H_c \arrow[d, "Hf"] \\
F_{c'} \arrow[r, "\alpha_{c'}"]           & G_{c'} \arrow[r, "\beta_{c'}"]           & H_{c'}             
	\end{tikzcd}
 \end{center}
	\item $\alpha T \colon FT \rightarrow GT \colon c' \mapsto (FT(c') \xrightarrow{\alpha_{T(c')}} GT(c'))$是一个自然变换
	\item $T\alpha \colon TF \rightarrow TG \colon c \mapsto (TF(c) \xrightarrow{T(\alpha_{c})} TG(c))$是一个自然变换
\end{itemize}
\end{lemma}


上面这个第一个结论是一个很显然的结论,两个交换图拼在一起还是一张交换图,所以自然变换之间是可以进行复合运算的,特殊地,每个函子$F$都存在一个自身到自身单位自然变换$1_F \colon F \rightarrow F$,其实除了上面第一个结论之外还有两个$\alpha T$和$T\alpha$函子和自然变换的复合我没有看懂,似乎domain和codamin都是函子复合。

如果$\mathcal{C}$和$\mathcal{D}$都是小范畴,则以范畴$\mathcal{C}$到范畴$\mathcal{D}$的所有函子为对象,以自然变换为态射可以形成一个范畴$[\mathcal{C},\mathcal{D}]$,称为$\textbf{函子范畴}$。

\begin{example}
设 $P \colon Set \rightarrow Set $是幂集函子,它和单位函子$1_{Set}$构成一个自然变换,因为下图交换
\begin{center}
% https://tikzcd.yichuanshen.de/#N4Igdg9gJgpgziAXAbVABwnAlgFyxMJZABgBpiBdUkANwEMAbAVxiRAEEQBfU9TXfIRQAmclVqMWbAAoAKdgEpuvEBmx4CRMsPH1mrRCABCyvusFFRO6nqmG5RpV3EwoAc3hFQAMwBOEAFskMhAcCCQARhtJAxAAHTiAIxgcOgB9Th4ffyDEELCkURAGOmSGaX4NIRBfLDcACxwQaP02b1MQP0DC6gLEAGYWu3iklPSTLM6cyN7wgaHYuW8nCi4gA
\begin{tikzcd}
A \arrow[rr, "\beta_A"] \arrow[dd, "f"'] &  & P(A) \arrow[dd, "P(f)"] \\
                                         &  &                         \\
B \arrow[rr, "\beta_B"]                  &  & P(B)                   
\end{tikzcd}
\end{center}
\end{example}
我在想这和函子$P$的定义有什么区别?注意这里$\beta_A$是集合元素之间的映射对任意的$x \in A$有$x \mapsto \{x\} \in P(A)$,而不是函子定义的对象映射,这里存在的自然变换是$\alpha \colon 1_{Set} \rightarrow P$,其中$1_{Set}$是范畴$Set$的单位函子。



\end{document}
