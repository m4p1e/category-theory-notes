\documentclass{article}
\usepackage{ctex}
\usepackage{tikz}
\usepackage{amsthm}
\usepackage{amsmath}
\usepackage{amssymb}
\usetikzlibrary{cd}



%theorem,corollary,lemma
\newtheorem{theorem}{Theorem}[section]
\newtheorem{corollary}{Corollary}[theorem]
\newtheorem{lemma}[theorem]{Lemma}
\newtheorem{definition}[theorem]{Definition}


\title{Category Theory}
\author{maple,菜鸡}
\date{\today}

\begin{document}
\maketitle

\section{Functoriality}

\begin{flushleft}
\textbf{函子}(functor)\\
\begin{definition}
一个函子$F \colon \mathcal{C} \rightarrow \mathcal{D}$由两个映射组成
\begin{itemize}
 \item 每一个对象$c \in \mathcal{C}$,对应一个对象$F_c \in \mathcal{D}$。
 \item 每一个态射$f \colon c \rightarrow	c' \in \mathcal{C}$,对应一个态射$Ff \colon Fc \rightarrow Fc' \in \mathcal{D}$,所有$Ff$的domain和codomian分别对应$f$的domain和codomain。 
\end{itemize}
\end{definition}
同时还满足两个公理(functoriality aximos)
\begin{itemize}
 	\item 对任意的可复合态射$f$,$g \in \mathcal{C}$,$Fg \cdot Ff = F(g \cdot f)$
 	\item 对任意的对象$c \in \mathcal{C}$,$F(1_c)=1_{F_c}$
\end{itemize}
通俗一点来说,函子由两个范畴之间对象的映射和态射的映射组成,同时保留了范畴所有的结构,包括domains,codomain,复合律,identity。
\end{flushleft}

\begin{lemma}
函子保留同构
\end{lemma}

\begin{flushleft}
\textbf{同变函子和逆变函子}(covariant and contravariant)\\
\begin{definition}
前面定义的函子也叫同变函子,那么一个逆变函子从$C$到$D$为$F: \mathcal{C}^{op} \rightarrow \mathcal{D} $,也是由两个映射组成
\begin{itemize}
	\item 每一个对象$c \in \mathcal{C}$,对应一个对象$F_c \in \mathcal{D}$。
	\item 每一个态射$f \colon c \rightarrow	c' \in \mathcal{C}$,对应一个态射$Ff \colon Fc' \rightarrow Fc \in \mathcal{D}$,所有$Ff$的domain和codomian分别对应$f$的codomain和domain
\end{itemize}
\end{definition}
同时也满足两个公理()\\
\begin{itemize}
	\item 对任意的可复合的$f$,$g \in \mathcal{C}$,$Ff \cdot Fg = F(g \cdot f)$
	\item 对任意的对象$c \in \mathcal{C}$,$F(1_c)=1_{F_c}$
\end{itemize}
相对于同变函子来说,$f$对应的像$Ff$的箭头反过来了,下面这个图可以很形象的描述	
\end{flushleft}

\begin{center}
$C^{op} \xrightarrow{F} \mathcal{D}$ \\
% https://tikzcd.yichuanshen.de/#N4Igdg9gJgpgziAXAbVABwnAlgFyxMJZABgBpiBdUkANwEMAbAVxiRAGMQBfU9TXfIRRkAjFVqMWbdgHJuvEBmx4CREeXH1mrRCABiAfU48+ywWtJjqWqbsPBZXbuJhQA5vCKgAZgCcIALZIZCA4EEgATNQ4dFgMbAF0aHBh8j7+QYjqoeGIAMzRsfG6icmpJiB+gcHRudk2OpVplRlIBTmR1pKNet7OXEA
\begin{tikzcd}
c \arrow[r, maps to] \arrow[d, "f"] & F_c                    \\
c' \arrow[r, maps to]               & F_{c'} \arrow[u, "Ff"]
\end{tikzcd}
\end{center}

\begin{flushleft}
\textbf{c-Functor} \\
%待补充
\begin{center}
$\mathcal{C} \xrightarrow{C(c,-)} \mathcal{S}et$ \\
$\mathcal{C}^{op} \xrightarrow{C(-,c)} \mathcal{S}et$ \\
\end{center}
\end{flushleft}


\begin{flushleft}
$\textbf{Product}$ \\
\begin{definition}
对任意的两个范畴 $\mathcal{C}$ 和 $\mathcal{D}$,他们的积是一个新范畴 $\mathcal{C} \times \mathcal{D}$\\
\begin{itemize}
	\item 对象是有序对 $\left( c,d \right)$,其中$c$是$\mathcal{C}$中的一个对象,$d$是是$\mathcal{D}$的一个对象 
	\item 态射也是有序对 $\left(f,g \right) \colon \left(c,d\right) \rightarrow \left(c',d'\right)$,$f\colon c \rightarrow c' \in \mathcal{C}$,$g \colon d \rightarrow d' \in \mathcal{D}$
\end{itemize}
\end{definition}
\end{flushleft}


\begin{flushleft}
\textbf{双函子的形式化定义}(bifunctor) \\ 
\begin{definition}
%https://encyclopediaofmath.org/wiki/Bifunctor
它叫双函子,不如叫二元函子,因为bifunctor是binary functor的缩写,对范畴$\mathcal{C}_1,\mathcal{C}_2,\mathcal{D}$,二元函子表示为:\\

\begin{center}
$F \colon \mathcal{C}_1 \times \mathcal{C}_2 \rightarrow \mathcal{D}$
\end{center}
\end{definition}

这个函子的domain是两个范畴的积,例如Abel群看做一个范畴$\mathbf{G}$,其中加法运算 $+ \colon \mathbf{G} \times \mathbf{G} \rightarrow \mathbf{G}$是一个双函子(证明过程证明functor满足的两个公理即可,保持identity和复合律)。 

\end{flushleft}


\begin{flushleft}
$\textbf{自然双函子}$,好吧它的真正名字叫$\textbf{射影函子}$ \\
\begin{center}
$P_{\mathcal{C}} \colon \mathcal{C} \times \mathcal{D} \rightarrow \mathcal{C},\left( A,B \right)\mapsto A,\left( f,g \right)\mapsto f$ \\ \vbox{}
$P_{\mathcal{D}} \colon \mathcal{C} \times \mathcal{D} \rightarrow \mathcal{D},\left( A,B \right)\mapsto B,\left( f,g \right)\mapsto g$ \\
\end{center} 
\end{flushleft}




\begin{flushleft}
\textbf{射影函子的万有性质} \\
对任意的范畴 $\varepsilon$ 及函子 $R\colon \varepsilon \rightarrow \mathcal{C}$    和 $T\colon \varepsilon \rightarrow \mathcal{D}$ ,存在唯一的函子 $F\colon \varepsilon \rightarrow \mathcal{C}\times\mathcal{D}$ 使得 $P_{\mathcal{C}}F=R,P_{\mathcal{D}}F=T$, 即下面交换图表示
\end{flushleft}

% https://tikzcd.yichuanshen.de/#N4Igdg9gJgpgziAXAbVABwnAlgFyxMJZABgBpiBdUkANwEMAbAVxiRAB12BbOnACwDGjYAGEAvgAJOeLvCndeg4QBExIMaXSZc+QigBM5KrUYs2nHvyENRajVux4CRQ-uP1mrRB3b0ATjBo2AzO9iAYjrpEZG7UHmbeForWwKrqxjBQAObwRKAAZn4QXEhkIDgQSACM1Ax0AEYwDAAK2k56IAww+TggcaZeIM0A+sBJVsLidpoghcWl1BVIAMz9nmwjYwoTNqrTBUUliIbllYirJuveACrqM3NHJ0uINZcJIABKdwfzx4tnZQYWDAgygdDgfEyfTegwAYukxEA
\begin{center}
\begin{tikzcd}
\mathcal{C} \times \mathcal{D} \arrow[rr, "P_{\mathcal{C}}"] \arrow[dd, "P_{\mathcal{D}}"] &  & \mathcal{C}                                                           \\
                                                                                           &  &                                                                       \\
\mathcal{D}                                                                                &  & \varepsilon \arrow[ll, "T"] \arrow[uu, "R"] \arrow[lluu, "F", dashed]
\end{tikzcd}
\end{center}

\section{自然变换}


\end{document}
