\documentclass{article}
\usepackage{ctex}
%\usepackage{math490}
\usepackage{tikz}
\usetikzlibrary{cd}
\usepackage{amsthm}
\usepackage{amsmath}
\usepackage{amssymb}
\usepackage{unicode-math} %为了花体数字
%\usepackage[bbgreekl]{mathbbol}
\usepackage[all,cmtip]{xy} %画交换图的package
\usepackage{xspace}
\usepackage{stmaryrd} %for double square bracket
\usepackage{proof}

\usepackage{hyperref} %url
\hypersetup{
    colorlinks=true,
    linkcolor=blue,
    filecolor=magenta,      
    urlcolor=cyan,
    pdftitle={Overleaf Example},
    pdfpagemode=FullScreen,
    }

\usepackage{titlesec}
\titleformat{\section}[block]{\color{blue}\Large\bfseries\filcenter}{}{1em}{}
\titleformat{\subsection}[hang]{\color{red}\Large\bfseries}{}{0em}{}
\usepackage[textwidth=18cm]{geometry} %页宽

%theorem,corollary,lemma
%\newtheorem{theorem}{Theorem}[section]
%\newtheorem{corollary}{Corollary}[theorem]
%\newtheorem{lemma}[theorem]{Lemma}
%\newtheorem{definition}[theorem]{Definition}
%\newtheorem{example}[theorem]{Example}

\newtheorem{theorem}{Theorem}[section]
\newtheorem{lemma}[theorem]{Lemma}
\newtheorem{corollary}[theorem]{Corollary}
\newtheorem{proposition}[theorem]{Proposition}
\newtheorem{example}[theorem]{Example}
\newtheorem{definition}[theorem]{Definition}
\newtheorem{remark}[theorem]{Remark}
\newtheorem{exercise}{Exercise}[section]
\newtheorem{annotation}[theorem]{Annotation}


\newcommand{\defeq}{\coloneqq}
\newcommand{\eqdef}{\eqqcolon}

% shortcuts for various fonts
\newcommand{\tbf}{\textbf}
\newcommand{\mcl}{\mathcal}
\newcommand{\mbf}{\mathbf}
\newcommand{\mbb}{\mathbb}
\newcommand{\msr}{\mathscr}
\newcommand{\mfr}{\mathfrak}
\newcommand{\msf}{\mathsf}
\newcommand{\mrm}{\mathrm}

% shortcuts for common categories
\newcommand*{\cat}[1]{\textsf{#1}\xspace}
\newcommand{\Cat}{\textsf{Cat}\xspace}
\newcommand{\CAT}{\textsf{CAT}\xspace}
\newcommand{\Set}{\textsf{Set}\xspace}
\newcommand{\Fin}{\textsf{Fin}\xspace}
\newcommand{\Top}{\textsf{Top}\xspace} % replaced logical Top ⫪
\newcommand{\Ring}{\textsf{Ring}\xspace}
\newcommand{\Field}{\textsf{Field}\xspace}
\newcommand{\Group}{\textsf{Group}\xspace}
\newcommand{\Monoid}{\textsf{Monoid}\xspace}
\newcommand{\Ab}{\textsf{Ab}\xspace}
\newcommand{\Rng}{\textsf{Rng}\xspace}
% these can only be used in math mode
% if you can come up with a macro that works in both tell me
\newcommand{\Setp}{\textsf{Set}_*}
\newcommand{\Finp}{\textsf{Fin}_*}
\newcommand{\Topp}{\textsf{Top}_*}
\newcommand*{\Mod}[1]{\textsf{Mod}_{#1}}
\newcommand*{\Mat}[1]{\textsf{Mat}_{#1}}
\newcommand*{\Vect}[1]{\textsf{Vect}_{#1}}

\newcommand{\zero}{\mbb{0}}
\newcommand{\one}{\mbb{1}}
\newcommand{\two}{\mbb{2}}
\newcommand{\three}{\mbb{3}}
\newcommand{\four}{\mbb{4}}
\newcommand{\five}{\mbb{5}}
\newcommand{\six}{\mbb{6}}
\newcommand{\seven}{\mbb{7}}
\newcommand{\eight}{\mbb{8}}
\newcommand{\nine}{\mbb{9}}

\newcommand{\Proj}{\text{\bf Proj}}
\newcommand{\Affine}{\text{\bf Affine}}

% blackboard bold shortcuts
\renewcommand{\AA}{\mathbb{A}} % replaced Angstrom symbol Å
\newcommand{\BB}{\mathbb{B}}
\newcommand{\CC}{\mathbb{C}}
\newcommand{\DD}{\mathbb{D}}
\newcommand{\FF}{\mathbb{F}}
\newcommand{\KK}{\mathbb{K}}
\newcommand{\NN}{\mathbb{N}}
\newcommand{\PP}{\mathbb{P}}
\newcommand{\QQ}{\mathbb{Q}}
\newcommand{\RR}{\mathbb{R}}
\renewcommand{\SS}{\mathbb{S}}
\newcommand{\ZZ}{\mathbb{Z}}

% sans serif shortcuts
\newcommand{\sA}{\mathsf{A}}
\newcommand{\sB}{\mathsf{B}}
\newcommand{\sC}{\mathsf{C}}
\newcommand{\sD}{\mathsf{D}}
\newcommand{\sE}{\mathsf{E}}
\newcommand{\sF}{\mathsf{F}}
\newcommand{\sG}{\mathsf{G}}
\newcommand{\sH}{\mathsf{H}}
\newcommand{\sJ}{\mathsf{J}}
\newcommand{\sP}{\mathsf{P}}

% script shortcuts -- NOT USED FOR CATEGORIES, USE mathsf
% note: cD, cH, cL, and cR were commands in an earlier version of xypic now
% considered obsolete, the file xyv2.tex still defines them, hence the
% renewcommand
\newcommand{\cA}{\mathcal{A}}
\newcommand{\cB}{\mathcal{B}}
\newcommand{\cC}{\mathcal{C}}
\newcommand{\cD}{\mathcal{D}} % see note
\newcommand{\cF}{\mathcal{F}}
\newcommand{\cG}{\mathcal{G}}
\newcommand{\cH}{\mathcal{H}} % see note
\newcommand{\cI}{\mathcal{I}}
\newcommand{\cJ}{\mathcal{J}}
\newcommand{\cP}{\mathcal{P}}
\newcommand{\cL}{\mathcal{L}} % see note
\newcommand{\cM}{\mathcal{M}}
\newcommand{\cN}{\mathcal{N}}
\newcommand{\cO}{\mathcal{O}}
\newcommand{\cS}{\mathcal{S}}
\newcommand{\cT}{\mathcal{T}}
\newcommand{\cV}{\mathcal{V}}
\newcommand{\cX}{\mathcal{X}}
\newcommand{\cY}{\mathcal{Y}}
\newcommand{\cZ}{\mathcal{Z}}

\newcommand{\mm}{\mathfrak{m}}
\newcommand{\pp}{\mathfrak{p}}
\newcommand{\qq}{\mathfrak{q}}
\newcommand{\ee}{\mathfrak{e}}

\newcommand{\bC}{{\bf C}}
\newcommand{\bD}{{\bf D}}

% shortcuts for various greek letters
\newcommand{\al}{\alpha}
\newcommand{\be}{\beta}
\newcommand{\gm}{\gamma}
\newcommand{\Gm}{\Gamma}
\newcommand{\de}{\delta}
\newcommand{\De}{\Delta}
\newcommand{\ep}{\epsilon}
\newcommand{\ze}{\zeta}
\renewcommand{\th}{\theta} % replaced thorn þ
\newcommand{\Th}{\Theta}
\newcommand{\io}{\iota}
\newcommand{\ka}{\kappa}
\newcommand{\lm}{\lambda}
\newcommand{\Lm}{\Lambda}
\newcommand{\sm}{\sigma}
\newcommand{\Sm}{\Sigma}
\newcommand{\om}{\omega}
\newcommand{\Om}{\Omega}

% for embedding text in display style math
\newcommand*{\qtextq}[1]{\quad\text{#1}\quad}
\newcommand*{\qtext}[1]{\quad\text{#1}}
\newcommand*{\textq}[1]{\text{#1}\quad}
\newcommand{\IN}{\text{ in }}
\newcommand{\AND}{\text{ and }}
\newcommand{\OR}{\text{ or }}
\newcommand{\ST}{\text{ such that }}
\newcommand{\IF}{\text{ if }}

% 1/<stuff> seems to be the most common fraction
\newcommand*{\recip}[1]{\frac{1}{#1}}
\newcommand*{\trecip}[1]{\tfrac{1}{#1}}
\newcommand*{\ipr}[2]{\left\langle{#1},{#2}\right\rangle}

\newcommand{\id}{1}
\newcommand{\op}{\textrm{op}}
\newcommand{\iso}{\textrm{iso}}
\newcommand{\nil}{\emptyset}
\newcommand{\by}{\times}
\newcommand{\emb}{\hookrightarrow}
\newcommand{\wto}{\rightharpoonup}
\newcommand{\bic}{\leftrightarrow}
\newcommand{\To}{\Rightarrow}
\newcommand*{\inv}[1]{{#1}^{-1}}

\newcommand{\Ztwo}{\ZZ\sqbraq{\sqrt{2}}}
\newcommand{\rtwo}{\sqrt{2}}
\newcommand*{\lst}[3][1]{{#2}_{#1},\dots,{#2}_{#3}}
\newcommand*{\fm}[3][i]{\{{#2}_{#1}\}_{{#1}\in{#3}}}

% function defintions
\newcommand*{\xfunc}[4]{{#2}\colon{#3}{#1}{#4}}
\newcommand*{\func}[3]{\xfunc{\to}{#1}{#2}{#3}}
\newcommand*{\Func}[3]{\xfunc{\To}{#1}{#2}{#3}}
\newcommand*{\mono}[3]{\xfunc{\rightarrowtail}{#1}{#2}{#3}}
\newcommand*{\afunc}[4]{\xfunc{\leftrightarrows}{#1}{#3}{#4}\noloc{#2}}
\newcommand*{\epic}[3]{\xfunc{\twoheadrightarrow}{#1}{#2}{#3}}
\newcommand*{\incl}[3]{\xfunc{\hookrightarrow}{#1}{#2}{#3}}
\newcommand*{\pfunc}[3]{\xfunc{\rightrightarrows}{#1}{#2}{#3}}
\newcommand*{\idfunc}[1]{\func{\id_{#1}}{#1}{#1}}
\newcommand*{\oper}[2]{\func{#1}{{#2}\by{#2}}{#2}}
\newcommand*{\nper}[3]{\func{#1}{{#2}^{#3}}{#2}}
\newcommand*{\maps}[3]{\xfunc{\mapsto}{#1}{#2}{#3}}
\newcommand*{\isom}[3]{\xfunc{\cong}{#1}{#2}{#3}}

\newcommand*{\con}[3]{{#1}\equiv{#2}\mod{#3}}
\newcommand*{\pcon}[3]{{#1}\equiv{#2}\pmod{#3}}

% auto resize delimiters
\newcommand*{\norm}[1]{\left\|{#1}\right\|}
\newcommand*{\abs}[1]{\left\lvert{#1}\right\rvert}
\newcommand*{\qty}[1]{\left({#1}\right)}
\newcommand*{\sqr}[1]{\left[{#1}\right]}
\newcommand*{\floor}[1]{\left\lfloor{#1}\right\rfloor}
\newcommand*{\ceil}[1]{\left\lceil{#1}\right\rceil}
\newcommand*{\angl}[1]{\left\langle{#1}\right\rangle}

\newcommand*{\basis}[2]{\left\{{#1}_1,\dots,{#1}_{#2}\right\}}
\newcommand*{\inlnmat}[1]{\left(\begin{smallmatrix}#1\end{smallmatrix}\right)}
\newcommand*{\dispmat}[1]{\begin{pmatrix}#1\end{pmatrix}}
\newcommand*{\adj}[1]{\!\sqr{#1}}
\newcommand*{\fld}[1]{\!\qty{#1}}
\newcommand*{\fps}[1]{\!\sqr{\!\sqr{#1}\!}}
\newcommand*{\lps}[1]{\!\qty{\!\qty{#1}\!}}

% this is the preferred way to make non-italic text for function and operator
% names; it automatically handles spacing
\DeclareMathOperator{\ob}{ob}
\DeclareMathOperator{\fin}{fin}
\DeclareMathOperator{\mor}{mor}
\DeclareMathOperator{\dom}{dom}
\DeclareMathOperator{\len}{len}
\DeclareMathOperator{\cod}{cod}
\DeclareMathOperator{\colim}{colim}
\DeclareMathOperator{\range}{range}
\DeclareMathOperator{\Hom}{Hom}
\DeclareMathOperator{\End}{End}
\DeclareMathOperator{\Sym}{Sym}
\DeclareMathOperator{\Aut}{Aut}
\DeclareMathOperator{\Spec}{Spec}
\DeclareMathOperator{\Gal}{Gal}
\DeclareMathOperator{\ID}{id}
\DeclareMathOperator{\sgn}{sgn}
\DeclareMathOperator{\Conj}{Conj}
\DeclareMathOperator{\Sect}{Sect}

\newcommand{\redt}[1]{\textcolor{red}{#1}}
\newcommand{\bluet}[1]{\textcolor{blue}{#1}}
\newcommand{\dbracket}[1]{\ensuremath{\left\llbracket\,\vcenter{\hbox{$#1$}}\,\right\rrbracket}}

\title{Categorical Semantics for STLC}
\author{maple}
\date{\today}

\begin{document}
\maketitle
\tableofcontents

\newpage
\section{First Try}
\subsection{Category of Baby Type Theory}

\begin{definition}
\rm A \emph{baby type system} with only atomic types and where typing context are singletons.
\end{definition}

\begin{definition}
\rm A interpretion of baby type system is category $\cat{C}_{bT}$ such that objects are intrepretions of types and morphism are intrepretions of term-in-context(sequent). 
\begin{itemize}
	\item \emph{objects}: $\dbracket{A}$ where $A$ is atomic type;
	\item \emph{morphism}: $\dbracket{\Gamma \vdash E:A}:\dbracket{\Gamma} \to \dbracket{A}$, abbreviated to $\dbracket{E}:\dbracket{\Gamma} \to \dbracket{\Gamma}$;
	\item \emph{identity}: $\dbracket{x:A \vdash x:A} = 1_{\dbracket{A}}:\dbracket{A} \to \dbracket{A}$, it corresponds to \[\infer{x:A \vdash x:A}{}~\textsc{Var}\]
	\item \emph{composition}: $\dbracket{E_2[y\to E_1]} = \dbracket{E_2} \circ \dbracket{E_1}:\dbracket{A} \to \dbracket{C}$, it corresponds to \[\infer{x:A \vdash E_2[y\to E_1]}{x:A \vdash E_1:B & y:B \vdash E_2:C}~{Sub}\]
	\item \emph{unit law}: \[
	\begin{gathered}
	\dbracket{x:A \vdash E[x \to x]} =   \dbracket{x:A \vdash E:B} \circ \dbracket{x:A \vdash x:A}\\
	\dbracket{x:A \vdash E[y \to E]} =   \dbracket{y:B \vdash y:B} \circ \dbracket{x:A \vdash E:B} 
	\end{gathered}\]
	\item \emph{associative law}: Given $x:A \vdash E_1:B, y:B \vdash E_2:C, z:C \vdash E_3:D$, we have\[\dbracket{E_3[z\to E_2[y\to E_1]]} = \dbracket{E_3[z\to E_2][y \to E_1]} : \dbracket{A} \to \dbracket{D}\].
\end{itemize}
\end{definition}

\newpage
\subsection{Category of Propositions and Derivations}

\begin{definition}
\rm A netural deduction system corresponds to category $C_{\textsc{Nd}}$ such that objects are interpretations of propositions and morphisms are interpretations of derivations. 
\begin{enumerate}
	\item \emph{morphism}: a morphism $\dbracket{\mathcal{D}}:\dbracket{\Gamma} \to \dbracket{A}$ corresponds \[\deduce[\mathcal{D}]{\infer{A}{}}{\infer{}{\Gamma}}\]
	\item \emph{identity}: $1_{\dbracket{A}}$ corresponds identity derivation $\infer{A}{A}$. 
\end{enumerate} 
\end{definition}

\newpage
\subsection{Unit Type and Void Type}

\begin{definition}
\rm The unit type is interpreted by the terminal object in categorical representation
$$
\dbracket{\top} := 1
$$
where $\top$ is actually unit type\footnote{Though $\top$ is described in subtyping} and $1$ represents terminal object. The introduction rule for unit type 
$$
\infer[\top I]{\Gamma \vdash *: \top}{}
$$
is interpreted by the unique map to terminal object
$$
\dbracket{*} := !(\dbracket{\Gamma}):\Gamma \to 1.
$$   
\end{definition}

\begin{definition}
\rm The void type is interpreted by the initial object in categorical representation
$$
\dbracket{\perp} := 0
$$
where $\perp$ is actually unit type and $1$ represents initial object. The elimintaion rule for void type 
\end{definition}


\newpage
\subsection{Truth and Falsehood}

\begin{definition}
\rm Logical propositional constant truth is interpreted by the terminal object in categorical representation
$$
\dbracket{\top} := 1
$$
where $1$ represents terminal object. The natural deduction introduction rule for truth
$$
\infer[\top I]{\top}{&&}
$$
is interpreted by the unique map to termninal object
$$
\dbracket{\top I} := !(\dbracket{A}):\dbracket{A} \to 1. 
$$
\end{definition}

\begin{definition}
\rm Logical propositional constant falsehood is interpreted by the initial object in categorical representation
$$
\dbracket{\perp} := 0
$$
where $0$ represents initial object. The natural deduction elimination rule for falsehood
$$
\infer[\perp E]{A}{\perp}
$$
is interpreted by the unique map from the initial object:
$$
\dbracket{\perp E} := \mbox{!`}(\dbracket{A}): 0 \to \dbracket{A}
$$
\end{definition}

\newpage
\begin{thebibliography}{00}
\bibitem{2016} Edward Morehouse. Basic Category Theory. OPLSS, 2016. \url{https://www.ioc.ee/~ed/research/notes/intro_categorical_semantics.pdf}
\bibitem{2015} Edward Morehouse. Basic Category Theory. OPLSS, 2015.
\end{thebibliography}
\end{document}